
\section{Aufgabe 1: Python zum Fliegen bringen}

\subsection{Aufgabe 1a}
Bringt das folgende Programm zum laufen und testet es
\inputpython{Aufgaben/Aufgabe1/Aufgabe1.py}{0}{10000}

\subsection{Aufgabe 1b}
Schreibt ein Programm (ähnlich dem in Aufgabe 1a), welches die Höhe h und den Radius r
eines Zylinders einliest und dann die gesamte Oberfläche und das Volumen des Zylinders berechnet.

\subsection{Aufgabe 1c}
Schreibt ein Programm, welches für eine positive ganze Zahl die Quersumme
berechnet. Die Quersumme von 1337 bespielsweise ist 14.

\newpage
\section{Aufgabe 2: Dinge mit Listen und Schleifen}
Schreibt ein Programm, das eine Folge von Komma-Zahlen einliest. Das Ende
der Zahlenfolge wird erkannt durch das erste Zeichen, welches keine Zahl (also z.B. ein Buchstabe)ist.
Das Programm soll dann für die eingelesenen Zahlen folgende Größen berechnen:
\begin{enumerate}
\item Die Anzahl der eingelesenen Zahlen,
\item Die Summe der eingelesenen Zahlen,
\item Das Maximum der eingelesenen Zahlen,
\item Das Minimum der eingelesenen Zahlen,
\item Den Mittelwert der eingelesenen Zahlen
\end{enumerate}

\newpage